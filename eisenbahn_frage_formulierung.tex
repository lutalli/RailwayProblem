\documentclass{article}
\usepackage[ngerman]{babel}
\usepackage{amsthm, amsmath, amsfonts, enumitem}

\theoremstyle{definition}
\newtheorem{defi}{Definition}
\newtheorem*{prob*}{Problem}

\newcommand{\lgang}{\mathsf{L}}
\newcommand{\rgang}{\mathsf{R}}
\newcommand{\hgang}{\mathsf{H}}
\newcommand{\connec}{\vdash}
\newcommand{\nconnec}{\nvdash}
\newcommand{\lrh}{\mathsf{LRH}}

\begin{document}
    \begin{defi}
	Jede \textbf{Weiche} wird von einem existierenden Ding identifiziert (einer natürlichen Zahl, einem Namen, einer Farbe, etc.). Eine Weiche wird normalerweise mit der Buchstabe $w$ bezeichnet.
    \end{defi}

    \begin{defi}
	Eine Weiche $w$ besitzt einen \textbf{Linkgleisgang} $\lgang(w)$, einen \textbf{Rechtgleisgang} $\rgang(w)$ und einen \textbf{Hauptgleisgang} $\hgang(w)$. Die Definition der Abbildungen $\lgang$, $\rgang$ bzw.\ $\hgang$ muss nur so gegeben werden, dass $\lgang(w)$, $\rgang(w)$ und $\hgang(w)$ für eine bestimmte Weiche $w$ immer drei einander unterscheidbare, eindeutige Werte ergeben. ``unterscheidbar'' heißt, dass es für jede Weiche $w$ gilt
	\[ \lgang(w)\neq\rgang(w) \quad\land\quad \rgang(w)\neq\hgang(w) \quad\land\quad \lgang(w)\neq\hgang(w) \]
	Und ``eindeutig'' heißt, dass es für jedes Weichenpaar $(w_1, w_2)$ mit $w_1\neq w_2$ gilt
	\[ \{\lgang(w_1), \rgang(w_1), \hgang(w_1)\}\cap\{\lgang(w_2), \rgang(w_2), \hgang(w_2)\} = \emptyset \]
	Man kann z.B.\ definieren
	\[ \lgang(w) := w + 0.1 \qquad \rgang(w) := w + 0.2 \qquad \hgang(w) := w + 0.3 \]
	oder \[ \lgang(w) := (w,0,0) \qquad \rgang(w) := (0,0,w) \qquad \hgang(w) := (0,w,0) \]
	Es ist nur wichtig, dass man unter allen Gleisgängen unterscheiden kann.

	Ein Gleisgang wird normalerweise mit der Buchstabe $g$ bezeichnet.

	Wir geben außerdem die folgenden Abkürzungen an:
	\begin{align*}
	    \lrh &:= \{\lgang,\rgang,\hgang\} \\
	    \lrh(w) &:= \{\lgang(w),\rgang(w),\hgang(w)\} \\
	    \lrh[W] &:= \bigcup_{w\in W}\lrh(w)
	\end{align*}
    \end{defi}

    \begin{defi}
	Sei $W$ eine Menge von Weichen und sei $G=\lrh[W]$ die Menge aller Gleisgänge. Sei weiterhin ${\connec}\subset G\times G$ eine \textit{symmetrische Relation} (d.h.\ $g_1\connec g_2$ folgt $g_2\connec g_1$), die Information über die Verbindungen zwischen den \textit{Gleisgängen} (nicht zwischen den \textit{Weichen}!) beinhaltet. Die Struktur
	\[ \mathcal{B}=(W,G,\lrh,\connec) \]
	heißt ein \textbf{Bahnsystem}, wenn sie die folgenden Kriterien erfüllt:
	\begin{enumerate}[label=\roman*., ref=(\roman*)]
	    \item Jeder Gleisgang wird \textit{genau zu einem} anderen Gleisgang verbunden.
		\label{Bd:i}

		Dieses Kriterium teilt sich in zwei Sinne auf:
		\begin{enumerate}[label*=\arabic*.]
		    \item Jeder Gleisgang wird \textit{mindestens} zu einem anderen Gleisgang verbunden. Das bedeutet
			\[ \forall g_1\in G\,\exists g_2\in G\ (g_1\connec g_2) \]
			Mit anderen Worten,
			\[ \mathrm{dom}(\connec) = G \]
			Da $\connec$ symmetrisch ist, gilt auch $\mathrm{ran}(\connec)=\mathrm{dom}(\connec)=G$.
		    \item Jeder Gleisgang wird \textit{maximal} zu einem anderen Gleisgang verbunden. Das bedeutet
			\[ \forall g_1,g_2,g_3\in G\,(g_1\connec g_2\land g_1\connec g_3 \to g_2=g_3) \]
			Äquivalenterweise sagt man, dass die Relation $\connec$ eine \textit{Injektion} (und damit eine \textit{Surjektion}) ist.
		\end{enumerate}

	    \item Keine Weiche wird isoliert. Das bedeutet
		\[ \forall w_1,w_2\in W\,(w_1\neq w_2 \to \exists\mathsf{X},\mathsf{Y}\in\lrh\ (\mathsf{X}(w_1)\connec\mathsf{Y}(w_2))) \]
		Da laut \ref{Bd:i} kein Gleisgang offen bleibt, kann man diese Bedingung auch so deuten: Keine Weiche führt zu sich; Zwischen je zwei Weichen existiert immer mindestens eine Verbindung.
	\end{enumerate}
	$\connec$ heißt dann das \textbf{Verbindungsschema} des Bahnsystems $\mathcal{B}$.

	Die \textit{Dimension} des Bahnsystems $\mathcal{B}$, $|\mathcal{B}|$, ist die Anzahl der Weichen, $|W|$. Die Dimension eines Bahnsystems ist immer eine \textit{gerade Zahl}. Denn jede Weiche hat drei Gleisgänge, und wenn es eine ungerade Anzahl von Weichen gäbe, gäbe es auch eine ungerade Anzahl von Gleisgängen. Es wäre dann unmöglich, dass jeder Gleisgang genau zu einem anderen Gleisgang könnte verbunden werden.

	Die Menge aller Bahnsysteme der Dimension $n$ ist mit $\mathfrak{B}_n$ bezeichnet.
    \end{defi}

    \begin{defi}
	Seien $\mathcal{B}_1 = (W_1,G_1,\lrh_1,\connec_1)$ und $\mathcal{B}_2 = (W_2,G_2,\lrh_2,\connec_2)$ zwei Bahsysteme derselben Dimension. $\mathcal{B}_1$ und $\mathcal{B}_2$ sind \textit{äquivalent} (in Symbolen: $\mathcal{B}_1\equiv\mathcal{B}_2$), wenn \textit{mindestens eine} von den folgenden Kriterien erfüllt wird:

	\begin{enumerate}[label=\roman*., ref=(\roman*)]
	    \item (Arbitratität der Benennung) $(G_1,\connec_1)$ ist \textit{isomorph} zu $(G_2,\connec_2)$. D.h.\ es existiert eine Surjektion $f:G_1\to G_2$, so dass es für jede $g_1\in G_1$ und jede $g_2\in G_2$ gilt
		\[ g_1\connec_1 g_2 \quad\to\quad f(g_1)\connec_2 f(g_2) \]
		Der Isomorphismus $f$ entspricht einer Vertauschung oder Umnennung der Weichen. Ein Bahnsystem bleibt also grundsätzlich unverändert, wenn nur die Nummerierungen der Weichen einander vertauscht oder umgenannt werden, während die Relationen dazwischen beibehalten werden. (Es ist egal, ob ich die Weichen jeweils $1$, $2$, $3$, $4$, oder $2$, $4$, $3$, $1$, oder Hund, Katze, Frosch, Spatz nenne.)
	    \item (Symmetrie der Weiche) Es gilt $W_1=W_2$ und $G_1=G_2$. Nenne jetzt $W:=W_1=W_2$ und $G:=G_1=G_2$. Der Unterschied zwischen $\mathcal{B}_1$ und $\mathcal{B}_2$ besteht nur darin, dass Linkgleisgänge mit Rechtgleisgängen vertauscht werden. Das heißt in Symbolen:
		\begin{align*}
		    \forall g\in G (\quad&{\connec_1}(g) = {\connec_2}(g) \\
		    \lor\quad &\exists w\in W(g \connec_1 \lgang(w) \land g \connec_2 \rgang(w)) \\
		    \lor\quad &\exists w\in W(g \connec_1 \rgang(w) \land g \connec_2 \lgang(w)))
		\end{align*}
	    \item (Transtivität) Es existiert ein Bahnsystem $\mathcal{B}_3$ mit $\mathcal{B}_1 \equiv \mathcal{B}_3$ und $\mathcal{B}_3 \equiv \mathcal{B}_2$.
	\end{enumerate}
    \end{defi}

    Es geht jetzt darum, alle Bahnsysteme unter der Definition der Äquivalenz zu klassizieren, vor allem dadurch, die Anzahl aller unterschiedlichen \textit{Äquivalenz-klassen} einer gegebenen Dimension zu bestimmen.

    \begin{defi}
	Die \textit{Äquivalenzklasse} von einem Bahnsystem $\mathcal{B}$, bezeichnet mit $[\mathcal{B}]_\equiv$ oder einfach $[\mathcal{B}]$, ist die Menge aller Bahnsysteme, die äquivalenz zu $\mathcal{B}$ sind:
	\[ [\mathcal{B}] = \{\mathcal{B}':\mathcal{B}\equiv\mathcal{B}'\} \]
    \end{defi}
    
    \begin{prob*}
	Sei $n$ eine gerade Zahl. Bestimme die Anzahl aller Elemente in der Menge
	\[ \{ [\mathcal{B}] : \mathcal{B} \in \mathfrak{B}_n \} \]
    \end{prob*}

\end{document}
